\chapter{Заключение}

В ходе работы построен алгоритм поиска сообществ в ориентированном взвешенном графе, основанный на гедонической игре. Алгоритм оказался достаточно хорошо интерпретируемым. Удалось понять, как он работает для ориентированных и неориентированных графов, чем отличаются устойчивые разбиения на сообщества и почему. Численно проанализирован порядок роста количества кластеров в эксперименте с контролируемой ориентацией ребер при весах вершин и ребер, равных константе. Проведены статистические оценки распределений по количеству кластеров в случае константных весов вершин и ребер, распределений по размерам кластеров в случае константных весов ребер и весах вершин, посчитаных с помощью degree-центральности. Установлено, что количество кластеров подчиняется нормальному распределению с параметрами, посчитанными методом максимального правдоподобия. Распределение по размерам кластеров похоже на распределение Пуассона, но статистические методы дают не слишком хорошие результаты. Проведено сравнение модулярности устойчивой кластеризации с максимально возможной для данной топологии, объяснено ее различие для центральностей betweenness и PageRank. Оценить порядок роста количества кластеров и теоретически оценить распределения пока не удалось. 