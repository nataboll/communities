\chapter{Введение}

Поиск сообществ в графах, рассматриваемый в данной работе, является одной из разновидностей задач кластеризации, относящейся к классу задач обучения без учителя. В данной задаче рассматривается граф $G(V, E)$, где $V=\{1,\dots,N\}$ - множество вершин, а множество упорядоченных пар $E=\{(u,v): u,v\in V\}$ - множество ребер. На множестве ребер задана функция $d:E\rightarrow [0,1]$, являющаяся функцией веса. Необходимо разбить множество вершин на группы так, чтобы полученное разбиение удовлетворяло заданным требованиям. Разбиение графов на сообщества отличается от задачи кластеризации точек в метрическом пространстве тем, что метрика в данной задаче не вводится, а поиск сообществ производится на основе топологии графа и, возможно, весов ребер. \\

Задача кластеризации графов может возникнуть в совершенно разных областях и преследовать разные цели. К примеру, известно, что общение между процессорами происходит гораздо медленнее по сравнению с передачей данных внутри процессора. В связи с этим, в компьютерных сетях межпроцессорное взаимодействие лучше минимизировать. Так как компьютерную сеть можно представить в виде неориентированного графа, задача сводится к тому, чтобы разбить вершины на группы так, что количество ребер между группами минимально. Поставленную задачу может решить алгоритм Ньюмана \cite{Newman2004}.\\

Важным примером задачи выявления сообществ в ориентированных графах является задача распознавания отмывания денег (money laundering) \cite{ml}. Для решения этой задачи ранее применялось обучение с учителем. На данный момент есть, например, алгоритм Louvain \cite{ml1}, который находит сообщества в ориентированном взвешенном графе.\\

На данный момент из существующих алгоритмов поиска сообществ в графах можно выделить две основные группы: 
\begin{enumerate}
	\item Алгоритмы, оптимизирующие некоторую целевую функцию, зависящую от заданного разбиения вершин на сообщества. Эту функцию обычно называют \textit{модулярностью}, ее определение будет дано в теоретическом введении.
	
	\item Теоретико-игровые алгоритмы, в которых для каждой вершины, называемой \textit{агентом}, вводится собственная \textit{функция прибыли}, которую агент старается оптимизировать. Требуется найти устойчивое разбиение агентов на сообщества, когда ни один агент не может увеличить свою прибыль за счет выбора другого сообщества.
\end{enumerate} 

В обеих группах есть и алгоритмы, работающие только для неориентированных графов, и универсальные алгоритмы, работающие для любых графов. Примером алгоритма, максимизирующего модулярность, является уже упомянутый алгоритм Ньюмана. Алгоритм Louvain тоже максимизирует функцию, называемую модулярностью, работает для ориентированных графов и не является игровым алгоритмом. Задача кластеризации ориентированных графов с помощью теории игр изучается в работе \cite{mining}. Теоретико-игровой алгоритм "label propagation"\ \cite{Raghavan} ищет сообщества в ориентированных графах, причем прибыль агента - количество его out-соседей, лежащих с ним в одном сообществе. Алгоритм InfoMap \cite{infomap}, кластеризующий ориентированные взвешенные графы, минимизирует среднюю длину случайного блуждания, закодированного с помощью кода Хаффмана. Существует также теоретико-игровой алгоритм, кластеризующий ориентированные графы, в котором функция прибыли агента пропорциональна центральности узла, соответствующего этому агенту, в полученном кластере \cite{Balliu}.\\

Среди теоретико-игровых алгоритмов кластеризации можно выделить \textit{гедонические} алгоритмы \cite{clusteringhg}. Строгое определение гедонической игры будет дано в теоретическом введении. Кратко, игра называется \textit{гедонической}, если для агента выбор фиксированной стратегии зависит от того, какие еще агенты выбрали эту стратегию \cite{framework}. На данный момент изучено несколько алгоритмов кластеризации неориентированных взвешенных графов \cite{clusteringhg} с помощью гедонических алгоритмов. Задача кластеризации ориентированного графа с помощью гедонических игр формулируется только в работах по теме "Fractional Hedonic Games"\ \cite{fhg2017}, но нигде не изучается. Гедонический алгоритм кластеризации ориентированного взвешенного графа может быть полезен, например, в упомянутой выше задаче распознавания отмывания денег. В связи с этим, целью данной работы является изучение кластеризации ориентированного взвешенного графа с помощью аппарата гедонических игр, построение соответствующего алгоритма и исследование его свойств, нахождение отличий равновесных разбиений на сообщества неориентированного и ориентированного графа. Также понять особенности полученного алгоритма может помочь сравнение с не гедоническими алгоритмами кластеризации, о которых шла речь выше.\\

Для тестирования алгоритма в работе использованы ансамбли случайных графов, сгенерированных с помощью моделей Эрдеша-Реньи и Барабаши-Альберт. Численная оценка полученных распределений производилась с помощью статистических методов, таких как проверка статистических гипотез, регрессия, QQ-график. Для оценки параметров распределений использовался метод максимального правдоподобия. Подробнее об используемых моделях и методах будет рассказано в теоретическом введении. 