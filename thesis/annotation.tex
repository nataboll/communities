\begin{center}
	\large 	Аннотация
\end{center}

Работа посвящена исследованию задачи поиска сообществ в ориентированных взвешенных графах с помощью гедонического алгоритма. На данный момент гедонические игры были использованы только для кластеризации неориентированных графов. Основная цель данной работы - понять, как может быть устроено разбиение на сообщества вершин ориентированного графа с помощью заданной на нем гедонической игры. В результате исследования построен алгоритм поиска сообществ, исследована зависимость равновесного разбиения от числа ориентированных ребер. Исследованы различия равновесного разбиения ориентированного и неориентированного графа при разных вероятностях в модели Эрдеша-Реньи. Часть результатов обоснована теоретически. Проведено сравнение модулярности результирующего разбиения с максимально возможной модулярностью для данной топологии. Проведен статистический анализ распределений по числу и размеру полученных кластеров, численно исследована зависимость параметров этих распределений от числа ориентированных ребер. Алгоритм протестирован на разных моделях случайных графов, построена зависимость характеристик структуры сообществ от параметров модели графа.