\chapter{Аннотация}

Работа посвящена исследованию задачи поиска сообществ в ориентированных взвешенных графах с помощью гедонического алгоритма. На данный момент гедонические игры были использованы только для кластеризации неориентированных графов. Основная цель данной работы - понять, как может быть устроено разбиение на сообщества вершин ориентированного графа. В результате исследования построен гедонический алгоритм поиска сообществ, исследована зависимость равновесного разбиения от числа ориентированных ребер. Часть результатов обоснована теоретически. Проведено сравнение полученного алгоритма с существующими алгоритмами кластеризации по модулярности и количеству кластеров в результирующем разбиении. Проведен статистический анализ распределений числа и размера полученных кластеров, численно исследована зависимость параметров этих распределений от числа ориентированных ребер. Алгоритм протестирован на нескольких моделях случайных графов, построена зависимость характеристик структуры сообществ от параметров модели графа.